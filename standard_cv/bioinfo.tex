% get document settings
%-------------------------------------
% LaTeX Resume for Software Engineers
% Author : Leslie Cheng
% License : MIT
%-------------------------------------

\documentclass[letterpaper,12pt]{article}[leftmargin=*]

\usepackage[empty]{fullpage}
\usepackage{enumitem}
\usepackage{ifxetex}
\ifxetex
  \usepackage{fontspec}
  \usepackage[xetex]{hyperref}
\else
  \usepackage[utf8]{inputenc}
  \usepackage[T1]{fontenc}
  \usepackage[pdftex]{hyperref}
\fi
\usepackage{fontawesome}
\usepackage[sfdefault,light]{FiraSans}
\usepackage{anyfontsize}
\usepackage{xcolor}
\usepackage{tabularx}
\usepackage{orcidlink}

%-------------------------------------------------- SETTINGS HERE --------------------------------------------------
% Header settings
\def \fullname {Jack Gisby}
\def \subtitle {}

\def \linkedinicon {\faLinkedin}
\def \linkedinlink {https://www.linkedin.com/in/jack-gisby-685407158/}
\def \linkedintext {/jack-gisby}

\def \phoneicon {\faPhone}
\def \phonetext {+44 7748 559964}

\def \emailicon {\faEnvelope}
\def \emaillink {mailto:me@jackgisby.com}
\def \emailtext {me@jackgisby.com}

\def \githubicon {\faGithub}
\def \githublink {https://github.com/jackgisby}
\def \githubtext {/jackgisby}

\def \headertype {\doublecol} % \singlecol or \doublecol

% Misc settings
\def \entryspacing {-0pt}

\def \bulletstyle {\faAngleRight}

% Define colours
\definecolor{primary}{HTML}{000000}
\definecolor{secondary}{HTML}{0D47A1}
\definecolor{accent}{HTML}{263238}
\definecolor{links}{HTML}{1565C0}

%------------------------------------------------------------------------------------------------------------------- 

% Defines to make listing easier
% \def \linkedin {\linkedinicon \hspace{3pt}\href{\linkedinlink}{\linkedintext}}
\def \phone {\phoneicon \hspace{3pt}{ \phonetext}}
\def \email {\emailicon \hspace{3pt}\href{\emaillink}{\emailtext}}
\def \github {\githubicon \hspace{3pt}\href{\githublink}{\githubtext}}
\def \website {\websiteicon \hspace{3pt}\href{\websitelink}{\websitetext}}
\def \orcid {\orcidlink{https://orcid.org/0000-0003-0511-8123} \hspace{0.3mm} \href{https://orcid.org/0000-0003-0511-8123}{orcid.org/0000-0003-0511-8123}}

% Adjust margins
\addtolength{\oddsidemargin}{-0.55in}
\addtolength{\evensidemargin}{-0.55in}
\addtolength{\textwidth}{1.1in}
\addtolength{\topmargin}{-0.6in}
\addtolength{\textheight}{1.1in}

% Define the link colours
\hypersetup{
    colorlinks=true,
    urlcolor=links,
}

% Set the margin alignment 
\raggedbottom
\raggedright
\setlength{\tabcolsep}{0in}

%-------------------------
% Custom commands

% Sections
\renewcommand{\section}[2]{\vspace{2pt}
  \colorbox{secondary}{\color{white}\raggedbottom\normalsize\textbf{{#1}{\hspace{7pt}#2}}}
  \vspace{0pt}
}

% Entry start and end, for spacing
\newcommand{\resumeEntryStart}{\begin{itemize}[leftmargin=2.5mm]}
\newcommand{\resumeEntryEnd}{\end{itemize}\vspace{\entryspacing}}

% Itemized list for the bullet points under an entry, if necessary
\newcommand{\resumeItemListStart}{\begin{itemize}[leftmargin=4.5mm]}
\newcommand{\resumeItemListEnd}{\end{itemize}}

% Resume item
\renewcommand{\labelitemii}{\bulletstyle}
\newcommand{\resumeItem}[1]{
  \item\small{
    {#1 \vspace{-2pt}}
  }
}

% Entry with title, subheading, date(s), and location
\newcommand{\resumeEntryTSDL}[4]{
  \vspace{-4pt}\item[]
    \begin{tabularx}{0.97\textwidth}{X@{\hspace{60pt}}r}
      \textbf{\color{primary}#1} & {\firabook\color{accent}\small#2} \\
      \textit{\color{accent}\small#3} & \textit{\color{accent}\small#4} \\
    \end{tabularx}\vspace{-6pt}
}

% Entry with title and date(s)
\newcommand{\resumeEntryTD}[2]{
  \vspace{-5pt}\item[]
    \begin{tabularx}{0.97\textwidth}{X@{\hspace{60pt}}r}
      \textbf{\color{primary}#1} & {\firabook\color{accent}\small#2} \\
    \end{tabularx}\vspace{-6pt}
}

% Entry for special (skills)
\newcommand{\resumeEntryS}[2]{
  \item[]\small{
    \textbf{\color{primary}#1 }{ #2 \vspace{-6pt}}
  }
}

% Double column header
\newcommand{\doublecol}[6]{
  \begin{tabularx}{\textwidth}{Xr}
    {
      \begin{tabular}[c]{l}
        \fontsize{35}{45}\selectfont{\color{primary}{{\textbf{\fullname}}}} \\
        {\textit{\subtitle}} % You could add a subtitle here
      \end{tabular}
    } & {
      \begin{tabular}[c]{l@{\hspace{1.5em}}l}
        {\small#4} & {\small#1} \\
        {\small#5} & {\small#2} \\
        {\small#6} & {\small#3}
      \end{tabular}
    }
  \end{tabularx}
}

% Single column header
\newcommand{\singlecol}[6]{
  \begin{tabularx}{\textwidth}{Xr}
    {
      \begin{tabular}[b]{l}
        \fontsize{35}{45}\selectfont{\color{primary}{{\textbf{\fullname}}}} \\
        {\textit{\subtitle}} % You could add a subtitle here
      \end{tabular}
    } & {
      \begin{tabular}[c]{l}
        {\small#1} \\
        {\small#2} \\
        {\small#3} \\
        {\small#4} \\
        {\small#5} \\
        {\small#6}
      \end{tabular}
    }
  \end{tabularx}
}


% determines what resume subtype to make
\newcommand{\resumeType}{bio}

\begin{document}

%---------------------------------------------------- HEADER ----------------------------------------------------

\headertype{\orcid}{\github}{}{\phone}{\email}{} % Set the order of items here
\vspace{-5pt} % Set a negative value to push the body up, and the opposite

\section{\faGears}{Statement}
 \vspace{3pt}
 \resumeEntryStart
  % {\small Computational biologist interested in the integration of -Omics datasets to investigate disease aetiology. Demonstrated ability to collaborate with technical and subject matter experts to extract valuable insights. Experienced in developing software and pipelines to process large biological datasets. Looking to learn and grow in a fast-paced environment where my work can have an immediate impact.\par}
  {\small Computational biologist with experience integrating large-scale multi-omic datasets to identify key biomarkers in clinical cohorts. Knowledgeable in the application of statistical methods and machine learning to -Omics measurements. Experienced in developing software and pipelines to process large biological datasets. Looking to learn and grow in a fast-paced environment where my work can have an immediate impact.\par}
 \resumeEntryEnd

\section{\faPieChart}{Experience}
    
\resumeEntryStart
    \resumeEntryTSDL
    {Immunology and Inflammation, Imperial College London}{August 2020 -- July 2023}
    {Doctoral Researcher}{London, UK}
    \resumeItemListStart
        \resumeItem {Integrated genetic, transcriptomic, proteomic and clinical data to investigate disease pathology.}
        \resumeItem {Applied longitudinal models, network analyses and machine learning to high-dimensional datasets.}
        \resumeItem {Processed and led the analysis of bulk RNA sequencing data generated for hundreds of samples. }
        % \resumeItem {Employed causal inference and colocalisation to understand disease-causing genetic variants.}
        \resumeItem {Employed causal inference and colocalisation to genetic data, identifying novel disease-causing targets.}
        \resumeItem {Used and developed Nextflow \& HPC-based pipelines to process sequence and variant data.}
        \resumeItem {Disseminated research findings through peer-reviewed publications and conference presentations.}
        \resumeItem {Delivered lectures and workshops, and created \href{https://github.com/ImperialCollegeLondon/ReCoDE_rnaseq_pipeline}{teaching materials for data science \faGithub}.}
    \resumeItemListEnd
    
    \resumeEntryTSDL
    {Biosciences, University of Birmingham}{July 2019 – July 2020}
    {Research volunteer}{Birmingham, UK}
    \resumeItemListStart
        \resumeItem {Developed an R/Bioconductor package for \href{https://bioconductor.org/packages/release/bioc/html/packFinder.html}{annotating rare transposons in genome sequences \faGithub}.}
        \resumeItem {Mined genome sequence data to uncover the impact of repetitive elements on the evolution of host genes.}
    \resumeItemListEnd

    \resumeEntryTSDL
    {Biosciences, University of Birmingham}{October 2019 – May 2020}
    {MSci Research}{Birmingham, UK}
    \resumeItemListStart
        \resumeItem {Created a \href{https://github.com/jackgisby/deepmet}{deep learning-based anomaly detection model \faGithub} that classifies molecular structures.}
        \resumeItem {Developed a \href{https://github.com/jackgisby/metaboblend}{Python package for processing and annotation of LC-MS metabolomics data \faGithub}.}
        \resumeItem {Adhered to software engineering best practices, including version control and automated testing.}
    \resumeItemListEnd

    \resumeEntryTSDL
    {The Binding Site}{Aug 2018 -- Aug 2019}
    {Industrial Placement}{Birmingham, UK}
    \resumeItemListStart
        \resumeItem {Took responsibility for the timely delivery of assay development projects and statistical reports.}
        \resumeItem {Communicated with, and delivered presentations to, a wide variety of interdisciplinary staff.}
    \resumeItemListEnd
\resumeEntryEnd

\section{\faGraduationCap}{Education}

  \resumeEntryStart
    \resumeEntryTSDL
      {Imperial College London}{Aug 2020 -- Aug 2023}
      {Computational Biology, PhD Candidate}{London, UK}
    %   Biomedical Data Science
  \vspace{6pt}
  
    \resumeEntryTSDL
      {University of Birmingham}{Sep 2016 -- Jul 2020}
      {Biochemistry with Professional Placement, MSci - First Class}{Birmingham, UK}
  \resumeEntryEnd

% \pagebreak

\section{\faGears}{Skills}
 \resumeEntryStart
  \resumeEntryS{Languages} {R, Python, Bash, SQL}
  \resumeEntryS{Packages} {pandas, multiprocessing, scikit-learn, PyTorch, caret, ggplot2, unittest, testthat}
  \resumeEntryS{Tools} {Git/GitHub, Docker, Conda, Nextflow, Airflow}  % Spark, Terraform
  \resumeEntryS{Compute} {High-performance computing clusters, exposure to the Google Cloud Platform}
  \resumeEntryS{Data} {Transcriptomics (bulk and single-cell RNA-seq), Proteomics (Olink and SomaLogic assays), Genomics (sequences and summary), Ontologies and Clinical data}
 \resumeEntryEnd

\vspace{5pt}

\section{\faFlask}{Selected Publications}

\resumeEntryStart
    % nature comms lrrc15
    \item \textbf{Jack S. Gisby}\textsuperscript{\textdagger}, Norzawani B. Buang\textsuperscript{\textdagger}, [...], David C. Thomas\textsuperscript{\textdagger}, James E. Peters\textsuperscript{\textdagger}. \textbf{Multi-omics identify falling LRRC15 as a COVID-19 severity marker and persistent pro-thrombotic signals in convalescence.} \textit{Nature Communications} 2022. \href{https://doi.org/10.1038/s41467-022-35454-4}{10.1038/s41467-022-35454-4}
      
    % elife proteomics
    \item \textbf{Jack S. Gisby}\textsuperscript{\textdagger}, Candice L Clarke\textsuperscript{\textdagger}, Nicholas Medjeral-Thomas\textsuperscript{\textdagger}, [...], Michelle Willicombe\textsuperscript{\textdagger}, David C Thomas\textsuperscript{\textdagger}, James E Peters\textsuperscript{\textdagger}. \textbf{Longitudinal proteomic profiling of dialysis patients with COVID-19 reveals markers of severity and predictors of death.} \textit{eLife} 2021. \href{https://doi.org/10.7554/eLife.64827}{10:e64827}
    
    % transposons 
    \item \textbf{Jack S. Gisby}, Marco Catoni. \textbf{The widespread nature of Pack-TYPE transposons reveals their importance for plant genome evolution.} \textit{PLOS Genetics} 2022. \href{https://doi.org/10.1371/journal.pgen.1010078}{10.1371/journal.pgen.1010078}
    
    % FAS preprint
    \item Lucija Klaric\textsuperscript{\textdagger}, \textbf{Jack S. Gisby}\textsuperscript{\textdagger}, Artemis Papadaki\textsuperscript{\textdagger}, [...], James F Wilson\textsuperscript{\textdagger}, James E Peters\textsuperscript{\textdagger}. \textbf{Mendelian randomisation identifies alternative splicing of the FAS death receptor as a mediator of severe COVID-19.} \textit{medRxiv} 2021. \href{https://doi.org/10.1101/2021.04.01.21254789 }{2021.04.01.21254789}
    
    \textdagger Equal contributions \hfill \orcidlink{https://orcid.org/0000-0003-0511-8123}{\hspace{0.3mm} \href{{https://orcid.org/0000-0003-0511-8123}}{orcid.org/0000-0003-0511-8123}}
    
\resumeEntryEnd

\section{\faBriefcase}{Other Projects}

  \resumeEntryStart

    \resumeEntryTD
      {A teaching resource for creating reproducible and shareable data pipelines}{\href{https://github.com/ImperialCollegeLondon/ReCoDE_rnaseq_pipeline}{GitHub}}
    \resumeItemListStart
      \resumeItem {Developed a resource to help teach students how to make reproducible pipelines for large datasets.}
      \resumeItem {Demonstrated Docker, Nextflow, version control, continuous integration and automated documentation.}
    \resumeItemListEnd

    \resumeEntryTD
      {A cloud ETL pipeline for combining TfL transport records with weather data}{\href{https://github.com/jackgisby/tfl-bikes-data-pipeline}{GitHub}}
    \resumeItemListStart
      \resumeItem {Used Airflow to schedule the ingestion and integration of multiple datasets on the Google Cloud Platform.}
      \resumeItem {Used PySpark to distribute the processing of datasets across a Dataproc cluster.}
      \resumeItem {Optimised an SQL (BigQuery) database using indexing and time-based partitioning.}
    \resumeItemListEnd

  \resumeEntryEnd

\end{document}
