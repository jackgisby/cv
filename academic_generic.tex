% Original Template Author: Rishi Shah

% Document Configuration
\documentclass{resume}
\usepackage[left=0.75in,top=0.6in,right=0.75in,bottom=0.6in]{geometry}
\newcommand{\tab}[1]{\hspace{.2667\textwidth}\rlap{#1}}
\newcommand{\itab}[1]{\hspace{0em}\rlap{#1}}
\name{Jack Gisby}
\address{Tel: (+44)7748 559964 \\ Email: j.gisby20@imperial.ac.uk \\ Site: jackgisby.com}

% Document Content 
\begin{document}
\begin{rSection}{Research Interests}
\vspace{1pt plus 1pt}
PhD candidate at Imperial College London interested in data-driven research and the application of -Omics technologies to investigate disease. I have the aspiration to pursue a career in biological research, exploiting the vast datasets produced by modern biology to answer fundamental research questions. My core project involves combining genomic, transcriptomic and proteomic datasets to identify molecular pathways that play a driving role in inflammatory disease. I am also interested in developing data processing and mining methods to facilitate the extraction of biological insights from -Omics datasets.

\end{rSection}
\begin{rSection}{Education}

\vspace{1pt plus 1pt}
{\bf Imperial College London} \hfill {\em July 2020 - August 2023} 
\\ PhD Candidate - Department of Immunology and Inflammation

\vspace{2pt plus 1pt minus 1pt}
\item Thesis: Using Multi-Omics to Understand Inflammatory Disease
\item Supervised by Dr James Peters and Dr Jacques Behmoaras 
\item Assessed by Professor Marc Chadeau and Dr Jessica Strid

\smallskip

{\bf University of Birmingham} \hfill {\em September 2016 - June 2020} 
\\ Biochemistry with Professional Placement (MSci) - First Class

\vspace{2pt plus 1pt minus 1pt}
\item Dissertation: A Prior Knowledge-Based Computational Workflow for \textit{de novo} Structural Elucidation of Small Molecules in Mass Spectrometry Metabolomics

\end{rSection}

% \begin{rSection}{Awards and grants}
% \end{rSection}

\begin{rSection}{Selected Publications}

\textbf{Longitudinal proteomic profiling of dialysis patients with COVID-19 reveals markers of severity and predictors of death} \\
First author \hfill  \textit{eLife} 2021 - doi:10.7554/eLife.64827

\vspace{2pt plus 1pt minus 1pt}
\item Lead the analysis of a high-dimensional proteomics (Olink) dataset with a complex cohort consisting of repeated measurements at inconsistent time points.
\item Applied linear mixed models and joint models to identify key proteins that changed over time following COVID-19 symptom onset.
\item Utilised supervised learning algorithms to identify biomarkers of severe disease.\\

\textbf{The widespread nature of Pack-TYPE transposons reveals their importance for plant genome evolution} \\
First author \hfill  bioRxiv 2021 - doi:10.1101/2021.06.18.448592 

\vspace{2pt plus 1pt minus 1pt}
\item Designed an algorithm for the specific annotation of repetitive elements that capture chromosomal DNA; implemented this as an R package available as part of the Bioconductor project.
\item Mined open source genetic data to demonstrate the abundance of these elements for multiple superfamilies and genomes. Found that recent insertions of these elements have impacted the evolution of genes. \\
 
\textbf{Mendelian randomisation identifies alternative splicing of the FAS death receptor as a mediator of severe COVID-19} \\
Equal first contributor \hfill  medRxiv 2021 - doi:10.1101/2021.04.01.21254789

\vspace{2pt plus 1pt minus 1pt}
\item Developed a pipeline to apply two sample Mendelian Randomisation comparing COVID-19 GWAS to Olink pQTLs. 
\item Used fine mapping and colocalisation tools in tandem with public data repositories to investigate causal variants of disease. \\

\textbf{Plasma Lectin Pathway Complement Proteins in Patients With COVID-19 and Renal Disease} \\
Equal first contributor \hfill  \textit{Frontiers in Immunology} 2021 - doi:10.3389/fimmu.2021.671052

\vspace{2pt plus 1pt minus 1pt}
\item Provided statistical support for the analysis of a repeated measures study design.
\item Generated data visualisations to investigate biological hypotheses. \\

\end{rSection}

\begin{rSection}{Conferences and presentations}

\item \textbf{Presentation - Biomarkers of the Future 2021} - Longitudinal proteomic profiling of dialysis patients with COVID-19 reveals markers of severity and predictors of death
\vspace{2pt plus 1pt minus 1pt}

\item \textbf{Poster - HUPO Reconnect 2021} - Longitudinal proteomic profiling of dialysis patients with COVID-19 reveals markers of severity and predictors of death
\vspace{2pt plus 1pt minus 1pt}

\item \textbf{Poster - UK-CIC Immunology 2021} - Longitudinal proteomic profiling of dialysis patients with COVID-19 reveals markers of severity and predictors of death
\vspace{2pt plus 1pt minus 1pt}

\item \textbf{Presentation - Longitudinal Studies 2021, Wellcome Genome Campus} - Longitudinal proteomic profiling of dialysis patients with COVID-19 reveals markers of severity and predictors of death
\vspace{2pt plus 1pt minus 1pt}

\end{rSection}

\begin{rSection}{Additional research experience}

\vspace{1pt plus 1pt}
\textbf{A Prior Knowledge-Based Computational Workflow for \textit{de novo} Structural Elucidation of Small Molecules in Mass Spectrometry Metabolomics} \\
Dissertation project, Dr Ralf Weber \hfill  \textit{October 2019 - June 2020}

\vspace{2pt plus 1pt minus 1pt}
\item Developed and optimised a Python package, \textit{Metaboblend}, to propose candidate molecules using tandem mass spectra (https://github.com/computational-metabolomics/metaboblend). 
\item Processed large, high-dimensional mass-spectrometry datasets for the validation of \textit{Metaboblend}. 
\item Used Python, SQLite and Bash to implement the \textit{Metaboblend} workflow on a Linux high-performance computing cluster for the generation of large structure databases. \\

\textbf{Development of an Immunoturbidimetric Assay for Serum Amyloid A for an Automated Clinical Analyser} \\
Industrial Placement, The Binding Site \hfill  \textit{August 2018 - August 2019}

\vspace{2pt plus 1pt minus 1pt}
\item Developed assays for inflammatory biomarkers, such as Serum Amyloid A, for use in clinical laboratories.
\item Interacted with, and regularly delivered presentations to, researchers from diverse fields and non-scientific staff to ensure the timely completion of my development project. 
\item Took responsibility for planning and carrying out experiments and statistical analyses in a research environment.

\end{rSection}

\begin{rSection}{Teaching experience}

\item \textbf{Lecture: RNA-seq in Molecular Epidemiology} - Developed a two-part lecture explaining the basics of RNA sequencing and its application to investigate genetic variants. Set homework and final exam questions for the module.
\vspace{2pt plus 1pt minus 1pt}

\item \textbf{Code Club Volunteer} - developed my teaching abilities and provided help with coding skills to 9-13 year-olds as part of the Code Club voluntary initiative.
\vspace{2pt plus 1pt minus 1pt}

\end{rSection}

\begin{rSection}{Specific Computational Skills}

\begin{tabular}{ @{} >{\bfseries}l @{\hspace{6ex}} l }
Operating Systems \ & Windows, Ubuntu \\
Programming Languages \ & Python, R, SQLite, Bash, working knowledge of C++ \\
Other Tools \ & Git, Linux HPC, Conda, Docker, Nextflow \\
General Software \ & MS Office, Latex  \\
\end{tabular}

\end{rSection}
\begin{rSection}{Additional Courses}

\item Systems biology: From large datasets to biological insight (1 week) by Wellcome Connecting Science and EMBL-EBI
\vspace{2pt plus 1pt minus 1pt}

\item Introduction to Assessment and Feedback by Imperial College London
\vspace{2pt plus 1pt minus 1pt}

\item Introduction to Teaching and Learning by Imperial College London
\vspace{2pt plus 1pt minus 1pt}

\item Profiling and optimisation in Python by Imperial College London
\vspace{2pt plus 1pt minus 1pt}

\item Computational Systems Biology: Deep Learning in the Life Sciences by MITx
\vspace{2pt plus 1pt minus 1pt}

\item Software Development with C++ (1 week) by University of Birmingham Research Computing
\vspace{2pt plus 1pt minus 1pt}

\item The Biostars Handbook: Bioinformatics Data Analysis by Istvan Albert
\vspace{2pt plus 1pt minus 1pt}

\item Introduction to Computer Science and Programming Using Python by MITx
\vspace{2pt plus 1pt minus 1pt}

\item Introduction to Computational Thinking and Data Science by MITx
\vspace{2pt plus 1pt minus 1pt}

\item Data Analysis for Life Sciences and Genomics Data Analysis by HarvardX

% \begin{rSection}{Additional publications}
% \end{rSection}

\end{rSection}
\end{document}
